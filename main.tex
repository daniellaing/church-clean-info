\documentclass[a4paper, 12pt, final, oneside]{article}

\title{Church Clean Information}
\author{Daniel Laing}
\date{\today}

\usepackage[margin=2.5cm]{geometry}
\usepackage[]{kpfonts}
\usepackage[super]{nth}
\usepackage{fancyhdr}
\usepackage{datetime}
\usepackage{hyperref}

\newcommand\version{}

\newcommand\email[1]{\href{mailto:#1}{\underline{#1}}}


\pagestyle{fancy}
\fancyhf{}
\renewcommand\headrulewidth{0pt}
\renewcommand\footrulewidth{0pt}
\fancyfoot[R]{\tiny\shortdate\today\hspace*{1em}\version}
\fancypagestyle{plain}{}


\begin{document}
\maketitle

\section*{Before you start}
\begin{description}
    \item[Coordinate a time with your team:] Use Church Suite to find their contact details. The church is usually empty
        on Monday nights. Check with \emph{Judy} at +44~7449~775270 or \email{administrator@bafreechurch.org.uk} what
        there are no other bookings at the time you want to clean.
    \item[Obtain a key:] If your team leader (or other team member) doesn't have a key, contact \emph{Daniel} at
        +44~7938~728200 or \email{church-clean@daniellaing.com}.
\end{description}

\section*{Cleaning equipment}
\begin{description}
    \item[Cleaning cupboards:] The upstairs cupboard is beside the big toilet in the foyer. The downstairs cupboard is
        at the bottom of the front/east stairs. Its key-box is next to the Sunday School cupboard in the corridor. The
        code for both boxes is \emph{1517}.
    \item[Hoovers:] A cordless Dyson hoover makes cleaning the sanctuary easier. Please return it to charge in the
        cupboard once you are done. The batteries for the downstairs hoovers are kept in the church flat. They are in
        the cupboard in the hall next to the kitchen. Please return them to charge once you are done.
    \item[Supplies:] If any supplies are missing, running low, or you use the last of, contact \emph{Anne Smith} at
        +44~7867~810525 or \email{donald.anne.smith@gmail.com}.
\end{description}

\section*{What to clean}
\begin{description}
    \item[Checklists:] Inside each cupboard are checklists which will guide you on what needs cleaning. They also serve
        as a record of what was cleaned when. It may not be appropriate to clean every item on the list every week.
    \item[Little Lambs:] During term time Little Lambs will clean the downstairs, so it will likely only need checked
        and a spot clean on any problem points.
    \item[Deacons:] The duty deacon will often empty the bins on a Sunday night.
\end{description}

\section*{Final checks}
\begin{itemize}
    \item[\(\square\)] Hoover (batteries) on charge.
    \item[\(\square\)] Empty hoovers of dust.
    \item[\(\square\)] Empty mop buckets.
    \item[\(\square\)] Empty bins.
\end{itemize}

\end{document}
